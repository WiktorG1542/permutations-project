\documentclass{article}
\usepackage{graphicx} % for including images

\usepackage{nopageno}
\begin{document}

%\title{Permutations}
%\author{\small Wiktor Grzywacz}
%\date{\small \today}
%\maketitle
\begin{center}
    \Large \textbf{Permutations - Wiktor Grzywacz}
\end{center}

\section{Explanation}
\par
The data shown on the graph is obtained by running the \textbf{run.sh} script located in the \textbf{input and output} directory.
For each pair of arguments you run the \textbf{makegraph} script with, one line with nine points is produced.
\par
Each one of the 9 points for any line on the graph, say 10 permutations of a set of 20, represents the time it took the \textbf{run.sh} script to execute 9 times,
each time for \textbf{x} pairs of 10 and 20 in the \textbf{input.txt} file, where \textbf{x} represents the number of the point.
\par
For example, if my generated graph has only a single line corresponding to 15 permutations of a set of 50 according to the legend, then
the first point of that line represents the time it took \textbf{run.sh} to run with 1, 15, 60 as its input, the second point 2, 15, 60, 15, 60 etc.

\section{Graph}

Here is a graph generated in gnuplot corresponding to the arguments given to the \textbf{makegraph} script:

\begin{figure}[h]
  \centering
  \includegraphics[width=\linewidth]{gnuplot_output.png}
\end{figure}

\end{document}
